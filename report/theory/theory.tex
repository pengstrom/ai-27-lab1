\section{Theory}

The A* algorithm is a type of best-first search using $f(n)$ as the estimation of the cheapest path through node $n$. The cost function is defined as
\[
f(n) = g(n) + h(n)
\]
where $g(n)$ is the cost of the path taken to node $n$ and $h(n)$ is a problem-specific heuristic estmating the cheapest path from node $n$ to the goal.

The algorithm is optimal given that $h$ is \emph{admissable} and \emph{consistent}. The latter is a stronger condition and implies the former.

\begin{description}
\item[Admissable] The function $h$ is \emph{admissable} if it never overstimates the cost to the goal.

\item[Consistent] The function $h$ is \emph{consistent} if $h(n) \leq d(n, n') + h(n')$ for every node $n$, $n'$ and edge $(n, n')$ where $d(n,n')$ is the cost of the action from $n$ to $n'$.
\end{description}

Actually, the tree-search variant of A* only requires admissability to ensure optimality. The graph-search variant still requires consistency.

For completeness, it is required that only a finite number of nodes with cost less than or equal to the optimal path cost.
