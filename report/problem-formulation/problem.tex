\begin{section}{The Delivery Man}
  To understand our solution and the algorithms that we have forcefully gladly chosen to implement, we must first explore the problem and study what it is we are trying to accomplish. This is done by a problem setting, describing the situation, a formalisation of the problem in terms of a game, which gives a mathematical structure of the problem, and a performance metric that contributes an order of solutions, which makes the comparable as ``better'' or ``worse''. This section is based on \cite{run} and \cite{lab}.
  
  \begin{subsection}{Problem Setting}
    From this point on, you are a delivery boy for the company ``Planet Express'', a man who has been sent to a small, two-dimensional city with a well-known rectangular street layout. The first task, which you have already accomplished, is to travel to intersection $(x, y)$ and get into a car. In the car, you have found a note which contains a list of pick-up and delivery points (all at intersections), $(x_{p_i}, y_{p_i}, x_{d_i}, y_{d_i})$, for $i\in \{1,2,3,4, 5\}$. 

    Your mission, according to the note, is to pick up and deliver all the packages in some order, without ever having more than one package in your car at a given time. It is also company policy, apparently, according to the note, that if you are at a pick-up point $(x_{p_i}, y_{p_i})$ you have to pick the package up. Similarly, the note clearly states that if company policy is every broken, dire consequences will be faced by someone. By you.
    
    You discover that the car doors are locked, and you note that the back side of the note says that the doors may be open once your work has been done. By you. Now, it seems that your company is completely unconcerned with your work-load, survival or well-being. However, \textbf{for you it is important to minimize time spent working}.

    Fortunately, the city's traffic control office provides continuous, up-to-date information about traffic load for each road-segment at each moment in time. So at each intersection, if you can develop some algorithm that is fast enough at computing, you will be able to determine the most optimal route at each intersection. Or, if you find it appropriate, you could wait a while at the intersection. It is up to you, it's not as if Hubert J. Farnsworth, the not so benevolent dictator of ``Planet Express'' has any idea of what is going on.
  \end{subsection}

  \begin{subsection}{Game Construction}
    We begin by constructing a ``formalized grid'' which will be the structure upon which we will build our game. 
    
    \begin{subsubsection}{The Game Graph}
    For the general problem, we begin by constructing a ``rectangular'' graph $G$. 

    Consider $M\times N$, where $M = {1, ..., m}$ and $N = \{1, ..., n\}$ for $m \geq 1, n \geq 1$. That is, each element $(i, j)$ of $M\times N$ can be represented by the intersection $(i, j)$ of horizontal line $i$ and vertical line $j$ in a rectangular grid with $M$ horizontal and $N$ vertical, lines.

    We denote the horizontal edges between node $(i, j)$ and $(i, j+1)$ in $M \times N$ by $h_{i,j}$.

    We denote the vertical edges between node $(i, j)$ and $(i+1, j)$ in $M \times N$ by $v_{i,j}$.

    For completeness (which will be apparent later), we also define $0_{i,j}$ as ``the edge'' from $(i,j)$ to $(i,j)$, which lacks geometrical interpretation in our visualization.

    Note that there are $M\cdot (N-1)$ horizontal edges, and $(M-1)\cdot N$ vertical edges and our construction corresponds visually to an $M$ by $N$ grid with evenly spaced horizontal and vertical lines.

    Let our graph $G$ have underlying vertices $V = M \times N$, and edge-set $E$ defined by the vertical and horizontal edges $h_{i,j}$ and $v_{i,j}$ as above. That is $E$ is formally defined by the binary relation
    \begin{equation}
      \begin{split}
      E = \{((i, j), (i, j+1)) &= h_{i,j}, 1 \leq i \leq m, 1 \leq j \leq (n-1)\}
      \\ \cup \{((i, j), (i+1, j)) &= v_{i,j}, 1 \leq i \leq (m-1), 1 \leq j \leq n \}
      \\ \cup \{((i, j), (i, j)) &= 0_{i,j}, 1 \leq i \leq m, 1 \leq j \leq n \}
      \end{split}
    \end{equation}

    Note that we have made fromal sense to the notion of nodes $(i, j)$, horizontal edges $h_{i,j}$ and vertical edges $v_{i,j}$ in terms of a graph. Similarly, it makes sense to talk about \textit{the grid of $G$}.

    \end{subsubsection}
    
    \begin{subsubsection}{The Game}
      TODO: Checka att det jag skriver överrensstämmer med hur programmet fungerar.

       For a graph $G$ constructed as above, you are given a starting point $(x, y)$ in its grid, and a set of packages $P = \{(p_i, d_i), i = 1, .., k\} \subseteq V\times V$. Furthermore, we play the following game:
      \begin{itemize}

        \item \textit{Objective}: Given a starting point $(x, y) \in V$, a set of packages $P = \{(p_i, d_i), i = 1, .., k\} \in V\times V$, your objective is to pick up each package at pick-up point $p_i$ and deliver it to corresponding delivery point $d_i$ by following the horizontal and vertical edges in $E$. That is, the \textbf{objective} is to deliver all the packages.

        \item \textit{Rules}:
          The game is played in turns. Each turn, you may take one of the following \textbf{actions} $A = \{a_1, ..., a_5 \}$ (depending on the to-node exists or not):
          \begin{enumerate}
          \item \textit{Up}: Move from node $(i, j)$ to node $(i+1, j)$.
          \item \textit{Down}: Move from node $(i, j)$ to node $(i-1, j)$.
          \item \textit{Left}: Move from node $(i, j)$ to node $(i, j-1)$.
          \item \textit{Right}: Move from node $(i, j)$ to node $(i, j+1)$.
          \item \textit{Stay}: Stay at node $(i, j)$.
          \end{enumerate}

          Turns are enumerated by increasing integers starting with $t = 1$

        \item \textit{Mechanics}:
          The mechanics are partly governed by the constraints.
          \begin{itemize}
          \item You arrive at a node $(i, j)$ at turn $t$ if the action you took at turn $t-1$ makes your position for turn $t$ node $(i, j)$.
            
          \item A package $(p, d)$ is \textbf{picked up} in turn $t$ if you arrive at node $p = (i, j)$ at turn $t$, and you do not hold a package when you arrive.

          \item You \textbf{hold} a package $p$ in turn $t$ if you've picked up $p$ in turn $t' \leq t$ and have not yet delivered the package. 

          \item A package $(p, d)$ is \textbf{delivered} in turn $t$ if you arrive at node $d = (i, j)$ at turn $t$ and you hold the package $(p, d)$.  

          \end{itemize}
          
        \item \textit{Mechanical constraints}:
          \begin{itemize}
          \item You do not pick up a package in the same turn that you deliver a package.
          \item You do not pick up a package at the start of the game.
          \item You can only hold one package at a time.
          \item You can never change packages.
          \item If you can pick up a package, you will.
          \item If you can deliver a package, you will.
          \item You can not pick up the same package twice.
          \end{itemize}
      \end{itemize}

      Furthermore, during each turn $t$, we have access to the following functions, which may depend both on the current turn $t$ or turns $t' \neq t$:
      \begin{itemize}
      \item A status function $s_t: P \rightarrow \{0,1,2 \}$, which determines one of the following statuses for each package on your delivery route:
        \begin{itemize}
        \item $s_t(p) = 0$ indicates that you have not picked up nor delivered the package.
        \item $s_t(p) = 1$ indicates that you have picked up (and consequently currently hold) the package.
          
        \item $s_t(p) = 2$ indicates that you have delivered package $p$.
        \end{itemize}

      \item A function $me_t: \emptyset \rightarrow M\times N$, where $me ()$ gives your location $(i, j)$ for your current turn.

      \item A cost function $c_t: E \rightarrow \mathbb{N}$ associated with each edge, and $c(0_{i,j}) = 0$. For example, if it is turn $3$ and we want to know the cost of the horizontal edge $h_{2,3}$ going from vertice $(2, 3)$ to vertice $(2, 4)$ at turn $3$, we call $c(h_{2,3})$.

      \item An edge function $e_t: A\times V \rightarrow E$ such that $e(a, (i,j)) = ((i, j),(i',j'))$ where $(i', j')$ is the node that results from taking action $a$ in position $(i,j)$ if the action is valid, and just $(i, j)$ otherwise.
      \end{itemize}
      
      A game \textbf{strategy} is \textit{how} and \textit{in what order} to deliver each package.

      A \textbf{solution} is a \textit{finite} sequence of actions $A_{seq} = (a_{i_1}, ..., a_{i_T})$, given in a \textit{finite} amount of time, associated with each turn that delivers all packages.

      Your \textbf{score} is determined by
      \begin{equation*}
        \sum_{t = 1}^T c_t(e_t(a_{i_t}, me_t()))
      \end{equation*}      

      \begin{paragraph}{Notes about the game.}
        By the mechanics, you may ``accidentally'' pick up a package that you would not want to pick up just yet, which may constrain \textit{how} you chose your route to the next pick-up destination. If the pick-up points for two packages $p_1, p_2$ coincide, there is no way to determine which will be picked up.
      \end{paragraph}

    \end{subsubsection}
    
    \begin{subsubsection}{Performance metric}
      The game above seems trivial if we were uninterrested in the score. For example, a finite solution for $k$ packages would be to just traverse the entire grid $2*k$ times, which would take a maximum of $m*n*(2*k+1)$ steps\footnote{To \textit{arrive} in a corner may take a maximum of $m*n$ steps, which ensures that traversing the entire graph twice delivers at least one package.}.

      But our interresting in the game is to minimize the cost of edge traversals while providing a tractable answer to the game.

      In particular, \textbf{we search for a solution $A_{seq}$ that is solution to the game above, and which minimizes cost function $c_t$ which may change dynamically with time}. That is, for a given edge $x\in E$, $c_t(x), c_{t'}(x)$ gives potentially different values when $t \neq t'$.
    \end{subsubsection}

    TODO icke-negativa edges är ett krav för A*?
  \end{subsection}

  

\end{section}
