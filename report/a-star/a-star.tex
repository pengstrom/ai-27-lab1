\section{A*}

Formulating the problem in terms of a game, with accompaigning well-defined problem formulation simplifies the explaination of the $A^*$-algorithm. Thta is, all functionality that is necessary to describe the problem has already been introduced. We assume that the reader is familiar with the concept of a priority queue $Q$, with functions $\texttt{Empty}(Q)$, $\texttt{Pop}(Q)$ and $\texttt{Insert}(n, Q)$ for an object $n$.

However, we have to assume that our cost function is independent of $t$, that is $c_t = c_{t'} =: c$ for all $t, t'$. This is later managed in our implementation.

Briefly, the $A^*$-algorithm uses a priority queue \textit{frontier} initialized to our current position, and the order of $n\in \textit{frontier}$ is determined by the path-cost of our current position to the node plus a heuristic $h(n)$. Then it expands the nodes in the order of \textit{frontier} and adds new reachable nodes to the frontier until a goal state is reached.

That is, unless the \textit{frontier} is empty, in which case there is no path to the goal. Note however, that this is not a scenario for our game.

Under certain conditions on the heuristic function $h$, the $A^*$ algorithm is optimal, which will be explained below.

\begin{subsection}{Heuristic function}

  A heuristic function $h(n)$ for a node $n$ is a function which estimates the ``cost of the cheapest path from the state at node $n$ to the goal'' \cite{rn}. The $A^*$ algorithm is optimal given that $h$ is \emph{admissable} and \emph{consistent}. The latter is a stronger condition and implies the former.

\begin{description}
\item[Admissable] The function $h$ is \emph{admissable} if it never overstimates the cost to the goal.

\item[Consistent] The function $h$ is \emph{consistent} if $h(n) \leq d(n, n') + h(n')$ for every state $n'$ immediately adjacent to $n$ (can be reached by one action) where $d(n,n')$ is the minimum cost of transitioning from $n$ to $n'$.
\end{description}

In particular, $d(n, n') = d((p,t,s), (p', t+1, s+c(p,p'))) = c(p,p')$.

\end{subsection}

\begin{subsection}{The $A^*$ algorithm}
  The precise formulation of the $A^*$-algorithm, which is the \textit{uniformed cost search} algorithm presented on page 84 in \cite{rn}, except that the node $n = (p, t, s)$ in the priority queue \textit{frontier} is ordered by $s + h(n)$ for a heuristic $h(n)$.

  There is one thing left to explain, namely that the algorithm maintains an \textit{explored} set, such that we do not add an already expanded node back into the \textit{frontier}.

  This is important because with the structure of the modified \textit{uniformed cost search} algorithm and by the definition of the heuristic function $h$, when we expand a node $n'$ for the first time, we are guaranteed that we have found the minimal cost to the node $n'$. This is what is known as \textbf{optimality} of the $A^*$ algorithm.

  In particular, the $A^*$ algorithm works both as a tree- and graph-search algorithm. Actually, the tree-search variant of A* only requires admissability of $h$ to ensure optimality. The graph-search variant still requires consistency of $h$ as well.

  For \textbf{completeness} of $A^*$, a guarantee that the algorithm actually find a solution when there is one \cite{rn}, it is required that only a finite number of nodes with cost less than or equal to the optimal path cost.

  
\end{subsection}

  \begin{subsection}{Choices for implementation}

    At first, we note that for dimension $n > 1, m > 1$ the game graph $G$ is not a tree, thus we will implement $A^*$ as a graph search algorithm. This requires a heuristic function $h$ which is both admissible and consistent.

    \begin{subsubsection}{Choice of heuristic $h$}
      Let's call an admissible and consitent heuristic $h$ acceptable, and we immediately note that $h(n) = 0$ is acceptable. This however doesn't really offer any help or improvements over the \textit{uniformed cost search} algorithm. So let's think some more.

      We se that $h(n) = \texttt{ManhattanDistance}(n, g) * \min_{e_1, e_2\in E}(c_t(e_1, e_2)$, that is, $h$ is the manhattan distance from $n$ to $g$ times the minimal edge cost, is an acceptable heuristic. To see this, note that $s + h(n) \leq d(n, n')$. In addition, it \textbf{punished moving further away from the goal node $g$}. In fact, it does so by making it less costly to move away from the goal when there are potential cheap roads on the longer route. Now we're actually improving upon the search.

      Further elaboration on what type of heuristic could be used might be fruitful in some cases, however as the problem does not expose how the edge costs are updated or assigned, we have not been capable of finding a better function. Thus we have chosen $h$ as immediately above.
    \end{subsubsection}
    
    \begin{subsubsection}{Implementation considerations}

      The goal is to provide create a function called \texttt{strategy} which can be passed to the game \texttt{runDeliveryMan} as the argument \texttt{carReady}. Since the function \texttt{runDeliveryMan} is provided in the R programming language, we have chosen to use R as well. 

      \textbf{A crucial piece} of the information available aboute \texttt{runDeliveryMan} is missing. We have no information about the non-negativity of the edge costs. Consider a graph with nodes $A, B, C$, and path costs $c(A, B) = -2, c(B, A) = 1, c(A, C) = 1$. If the goal is to go from $A$ to $C$, we easily see that the $A^*$-algorithm will never terminate. However, manual experimentation gives strong indications that the edge costs of \texttt{runDeliveryMan} are non-negative. Thus we can use it to find a lowest cost path. 

      \textbf{The dimension} of the physical problem to solve, rather than the theoretical, is low. The dimension of the modelled game is $(10, 10)$ and the number of packages to deliver is $5$. This makes it possible to disregard most normal consierations such as the search- and insert-complexity of queues, which favors a more direct programming style, which may not be tractable given larger dimensions. This has the advantage that the written program itself is easier to read for others. 
    \end{subsubsection}

    \textbf{The mechanics} of \texttt{runDeliveryMan} follows the mechanics for the game presented above. Therefor it is a direct consequence that the program we've written is guided by those. 

    \begin{subsubsection}{Implementation of $A^*$ }

      R provides a rather functional style of programming, and since the dimensionality of the problem permits rather unsofisticated algorithms, the actual implementation of the $A^*$ algorithm follows the algorithm presented in \cite{rn} as the \textit{uniformed cost search}.

    \end{subsubsection}

  \end{subsection}