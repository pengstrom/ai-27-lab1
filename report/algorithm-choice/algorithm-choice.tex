\begin{section}{Algorithm choice}
  We begin by turning our game above into a well-defined problem as defined in \cite{rn}. In order, we define the different components. So suppose that we are given a game with the strutcure presented above. Then let the tuple $(p, t, s)$ denote $(position, turn, score)$ of the game, which will be used to define our \textbf{states}.

  \begin{itemize}
  \item \textbf{Initial state.} The initial state is given by the tuple $(me(1), 1, 0)$.

  \item \textbf{Actions.} The actions for a given state $(p, t, s)$ are given by $A$.

  \item \textbf{Transition model.} We use $e_t$ as the \textit{transition function}, and thus our transition model \texttt{Result} is defined by
    \begin{equation*}
      \begin{split}
        \texttt{Result}( (p, t, s), a ) &= ( e_t(a, (i,j)),
        t+1, s+c_t(e_t(a, p))).
      \end{split}
    \end{equation*}

    
  \item \textbf{Successor.} We let $S(((i, j), t, s) = \{\texttt{Result}(((i,j), t, s), a), a\in A \}$ be the successors of a vertice $(i, j)$.

  \item \textbf{Goal.} We have reached our goal in state $(p, t, c)$ if $s_t(p') = 2$ for all $p' \in P$.

  \item \textbf{Path cost.} The path cost $c(((i,j),t,s), a, ((i',j'),t',s') = s'-s$ for any two valid states ($t' = t+1$ and $(i', j')$ is or is adjacent to $(i, j)$ in $G$). 
\end{itemize}
We note that we have stated our objective of the game in terms of a \textbf{well-formed problem} as per \cite{rn}. In particular, a \textbf{solution}, an action sequence leading from the inital state to a goal state, of the problem also gives a solution to our game, and an \textbf{optimal solution}, a solution which minimizes path cost, gives a searched for solution $A_{seq}$ as mentioned in the previous section. 

Futhermore, we define:
\begin{itemize}
  
\item \textbf{Valid states.} The valid states can be constructed as 
  $S_{valid} = \cup_{k\in \mathbb{N}}S_k$ where $S_0$ is the initial state and inductively we define
  \begin{equation*}
    S_k = S_{k-1}\cup \{\texttt{Result}((p,t,s),a), (p,t,s)\in S_{k-1}, a\in A \}.
  \end{equation*}
\end{itemize}

Now, from \cite{rn} we choose the $A*$-algorithm, as an informed search strategy, which, given a specific cost function as will be explained below, provides an \textbf{optimal} algorithm to solve the well-posed problem here. 1
\end{section}